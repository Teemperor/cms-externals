\section {HepPID headers}
\label{PIDheaders}

\subsection {ParticleIDTranslations.hh}

\begin{tabbing}

{\bf namespace HepPID} \\  \\

{\bf Free functions:} \\
\hspace{0.5in} {\bf int translateHerwigtoPDT( const int herwigID);} \\
\hspace{0.5in} {\bf int translatePDTtoHerwig( const int pid );} \\ 
\hspace{0.5in} {\bf void  writeHerwigTranslation( std::ostream \& os );} \\  \\

\hspace{0.5in} {\bf int translateIsajettoPDT( const int isajetID );} \\
\hspace{0.5in} {\bf int translatePDTtoIsajet( const int pid );} \\ 
\hspace{0.5in} {\bf void  writeIsajetTranslation( std::ostream \& os );} \\  \\

\hspace{0.5in} {\bf int translatePythiatoPDT( const int pythiaID );} \\
\hspace{0.5in} {\bf int translatePDTtoPythia( const int pid );} \\ 
\hspace{0.5in} {\bf void  writePythiaTranslation( std::ostream \& os );} \\  \\

\hspace{0.5in} {\bf int translateEvtGentoPDT( const int evtGenID );} \\
\hspace{0.5in} {\bf int translatePDTtoEvtGen( const int pid );} \\
\hspace{0.5in} {\bf void  writeEvtGenTranslation( std::ostream \& os );} \\  \\

\hspace{0.5in} {\bf int translatePDGtabletoPDT( const int pdgID);} \\
\hspace{0.5in} {\bf int translatePDTtoPDGtable( const int pid );} \\
\hspace{0.5in} {\bf void  writePDGTranslation( std::ostream \& os )} \\  \\

\hspace{0.5in} {\bf int translateQQtoPDT( const int qqID);} \\
\hspace{0.5in} {\bf int translatePDTtoQQ( const int pid );} \\
\hspace{0.5in} {\bf int translateQQbar( const int qqID);} \\
\hspace{0.5in} {\bf int translateInverseQQbar( const int pid );} \\
\hspace{0.5in} {\bf void  writeQQTranslation( std::ostream \& os );} \\  \\

\hspace{0.5in} {\bf int translateGeanttoPDT( const int geantID);} \\
\hspace{0.5in} {\bf int translatePDTtoGeant( const int pid );} \\  \\

\end{tabbing}

The writeXXXTranslation functions write a list of all known particle ID 
translations for the specified Monte Carlo generator.

QQ needs extra tranlation methods for the quark pair pseudo-particles since the 
ID numbers overlap.

\vfill\eject

\subsection {ParticleName.hh}

\begin{tabbing}

{\bf namespace HepPID} \\  \\

{\bf Free functions:} \\

\hspace{0.5in} {\bf std::string  particleName( const int );} \\
\hspace{1.0in}  Returns the HepPID standard name. \\

\hspace{0.5in} {\bf int  particleName( const std::string \& );} \\
\hspace{1.0in}  Returns the HepPID standard ID. \\

\hspace{0.5in} {\bf void  listParticleNames( std::ostream \& os );} \\
\hspace{1.0in} List all defined names.  \\

\hspace{0.5in} {\bf bool validParticleName( const int );} \\
\hspace{1.0in} Verify that this particle ID has a valid name.  \\

\hspace{0.5in} {\bf bool validParticleName( const std::string \& );} \\
\hspace{1.0in} Verify that this particle string has a valid ID.  \\

\hspace{0.5in} {\bf class ParticleNameMap;} \\
\hspace{0.5in} {\bf ParticleNameMap const \&  getParticleNameMap();} \\
\hspace{1.0in} Access ParticleNameMap for other purposes.  \\
\end{tabbing}

Only getParticleNameMap is allowed to access ParticleNameMap.
ParticleNameMap is initalized by the first call to getParticleNameMap.
Because the class is static, this initialization only happens once.
We use a data table so that compile time is not impacted.

\subsection {ParticleIDMethods.hh}

\begin{tabbing}

{\bf namespace HepPID} \\  \\

{\bf Free functions:} \\

\hspace{0.5in} {\bf unsigned short digit( location loc, const int \& );} \\
\hspace{1.0in} Return the digit at a named location in pid.  \\

\hspace{0.5in} {\bf int A(const int \& );} \\
\hspace{1.0in} If this is an ion, return A.  \\

\hspace{0.5in} {\bf int Z(const int \& );} \\
\hspace{1.0in} If this is an ion, return Z.  \\

\hspace{0.5in} {\bf int lambda( const int \& );} \\
\hspace{1.0in} If this is an ion, return nLambda.  \\

\hspace{0.5in} {\bf int           abspid( const int \& );} \\
\hspace{1.0in} Return the absolute value of the particle ID.  \\

\hspace{0.5in} {\bf int    fundamentalID( const int \& );} \\
\hspace{1.0in} Extract fundamental ID (1-100) if this is a "fundamental" particle.  \\

\hspace{0.5in} {\bf bool hasFundamentalAnti( const int \& );} \\
\hspace{1.0in} If this is a fundamental particle, does it have a valid antiparticle?  \\

\hspace{0.5in} {\bf int extraBits( const int \& );} \\
\hspace{1.0in} Returns everything beyond the 7th digit.  Mostly for internal use.  \\

\hspace{0.5in} {\bf bool isValid( const int \& );} \\
\hspace{1.0in}  Is this particle ID valid? \\

\hspace{0.5in} {\bf bool isMeson( const int \& );} \\
\hspace{0.5in} {\bf bool isBaryon( const int \& );} \\
\hspace{0.5in} {\bf bool isDiQuark( const int \& );} \\
\hspace{0.5in} {\bf bool isLepton( const int \& );} \\
\hspace{0.5in} {\bf bool isHadron( const int \& );} \\
\hspace{0.5in} {\bf bool isNucleus( const int \& );} \\
\hspace{0.5in} {\bf bool isPentaquark( const int \& );} \\
\hspace{0.5in} {\bf bool isSUSY( const int \& );} \\
\hspace{0.5in} {\bf bool isRhadron( const int \& );} \\
\hspace{0.5in} {\bf bool isDyon( const int \& );} \\
\hspace{0.5in} {\bf bool isQBall( const int \& );} \\
\hspace{1.0in} Is this a valid particle ID for the named particle type.  \\

\hspace{0.5in} {\bf bool hasUp( const int \& );} \\
\hspace{0.5in} {\bf bool hasDown( const int \& );} \\
\hspace{0.5in} {\bf bool hasStrange( const int \& );} \\
\hspace{0.5in} {\bf bool hasCharm( const int \& );} \\
\hspace{0.5in} {\bf bool hasBottom( const int \& );} \\
\hspace{0.5in} {\bf bool hasTop( const int \& );} \\
\hspace{1.0in} Does this particle contain the named quark? \\

\hspace{0.5in} {\bf int  jSpin( const int \& );} \\
\hspace{1.0in}  Returns 2J+1, where J is the total spin. \\

\hspace{0.5in} {\bf int  sSpin( const int \& );} \\
\hspace{1.0in} Returns 2S+1, where S is the spin.  \\

\hspace{0.5in} {\bf int  lSpin( const int \& );} \\
\hspace{1.0in} Returns 2L+1, where L is the orbital angular momentum.  \\

\hspace{0.5in} {\bf int threeCharge( const int \& );} \\
\hspace{1.0in} Return 3 times the charge.  If this is a Q-ball, return 30 times the charge.  \\

\hspace{0.5in} {\bf double charge( const int \& );} \\
\hspace{1.0in} Return the actual charge.  \\

\end{tabbing}

\vfill\eject
