This manual is a users' guide to using the Rivet generator validation
system. Rivet is a C++ class library, which provides the infrastructure and
calculational tools for particle-level analyses for high energy collider
experiments, enabling physicists to
validate event generator models and tunings with minimal effort and maximum
portability. Rivet is designed to scale effectively to large numbers of analyses
for truly global validation, by transparent use of an automated result caching
system.

The Rivet ethos, if it may be expressed succinctly, is that user analysis code
should be extremely clean and easy to write --- ideally it should be
sufficiently self-explanatory to in itself be a reference to the experimental
analysis algorithm --- without sacrificing power or extensibility. The machinery
to make this possible is intentionally hidden from the view of all but the most
prying users. Generator independence is explicitly required by virtue of all
analyses operating on the generic ``HepMC'' event record.

The simplest way to use Rivet is via the \kbd{rivet} command line tool, which
analyses textual HepMC event records as they are generated and produces output
distributions in a structured textual format. The input events are generated
using the generator's own steering program, which is typically provided for
modern codes; for older generators which have no default way to produce HepMC
output, the AGILe generator interface library and \kbd{agile-runmc} command may
be useful. For those who wish to embed their analyses in some larger framework,
Rivet can also be used as a library to run programmatically on HepMC event
objects with no special executable being required.

Before we get started, a declaration of intent: this manual is intended to be a
guide to using Rivet, rather than a comprehensive and painstakingly maintained
reference to the application programming interface (API) of the Rivet
library. For that purpose the online documentation at
\url{http://rivet.hepforge.org} should be sufficient -- in case of confusion
please contact the authors at \url{rivet@projects.hepforge.org}. Similar API
documentation is maintained for AGILe at \url{http://agile.hepforge.org}.


\subsection{Typographic conventions}
As is normal in computer user manuals, the typography in this manual is used to
indicate whether we are describing source code elements, commands to be run in a
terminal, the output of a command etc.

The main such clue will be the use of \kbd{typewriter-style} text: this
indicates the name of a command or code element --- class names, function names
etc. Typewriter font is also used for commands to be run in a terminal, but in
this case it will be prefixed by a dollar sign, as in \inp{echo "Hello" |
  cat}.  The output of such a command on the terminal will be typeset in
\outp{sans-serif} font. When we are documenting a code feature in detail (which
is not the main point of this manual), we will use square brackets to indicate
optional arguments, and italic font between angle brackets to represent an
argument name which should be replaced by a value,
e.g. \code{Event::applyProjection(\val{proj})}.

% Following the example of Donald Knuth in his books on \TeX{}, in this document
% we will indicate paragraphs of particular technicality or esoteric nature with a
% ``dangerous bend''\marginpar{\bendimg\\Dangerous bend} sign. These will
% typically describe internals of Rivet of which most people will be fortunate
% enough to remain happily ignorant without adverse effects. However they may be
% of interest to detail obsessives, the inordinately curious and Rivet
% hackers. You can certainly skip them on a first reading. Similarly, you may see
% double bend signs \marginpar{\dblbendimg\\Double bend} --- the same rules apply
% for these, but even more strongly.
